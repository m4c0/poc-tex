% Rewritting the following example to be compatible with an empty TeX format:
% https://tex.stackexchange.com/a/6493

\catcode`\{=1
\catcode`\}=2

\mag 1200

\parfillskip=0pt plus 1fil
\hsize=6.5truein
\vsize=8.7truein

\baselineskip=12pt

\countdef\pagenumber=0
\pagenumber=1

\output {
  \shipout\vbox {
    \vbox to \vsize {
      \unvbox 255
      \hbox to \hsize {
        \footsc The electronic journal of combinatorics {\footbf 16}, #R00
        \hss
        \footrm Page \number\pagenumber
      }
    }
  }
  \global\advance \pagenumber 1
}

\font\normal=cmr10
\font\big=cmr12 at 14pt
\font\footsc=cmcsc10 at 8truept
\font\footbf=cmbx10 at 8truept
\font\footrm=cmr10 at 10truept
\font\bold=cmbx10
\font\fixed=cmtt10

\fam 0pt \normal

\hbox to \hsize {\hss \big An elementary proof of the reconstruction conjecture\hss}

\vskip 12pt plus 4pt minus 4pt
\vskip 12pt plus 4pt minus 4pt

\hbox to \hsize {\hss D. Remifa \hss}
%\centerline{D. Remifa\footnote*{Thanks to
%  the editors of this wonderful journal!}}
\vskip 3pt plus 1pt minus 1pt
\hbox to \hsize {\hss Department of Inconsequential Studies \hss}
\hbox to \hsize {\hss Solatido College, North Kentucky, USA \hss}
\hbox to \hsize {\hss \fixed remifa@dis.solatido.edu \hss}

\vskip 12pt plus 4pt minus 4pt

\hbox to \hsize {\hss \footrm 
  Submitted: Jan 1, 2009; Accepted: Jan 2, 2009; Published: Jan 3, 2009
\hss}
\hbox to \hsize {\hss \footrm Mathematics Subject Classifications: 05C88, 05C89 \hss}

\vskip 12pt plus 4pt minus 4pt
\vskip 12pt plus 4pt minus 4pt

\hbox to \hsize {\hss \bold Abstract \hss}
\vskip 3pt plus 1pt minus 1pt
{
\advance \leftskip  20pt
\advance \rightskip 20pt
\noindent
  The reconstruction conjecture states that the multiset of unlabeled
  vertex-deleted subgraphs of a graph determines the graph, provided it
  has at least 3 vertices.  A version of the problem was first stated
  by Stanislaw Ulam.  In this paper, we show that the conjecture can
  be proved by elementary methods.  It is only necessary to integrate
  the Lenkle potential of the Broglington manifold over the quantum
  supervacillatory measure in order to reduce the set of possible
  counterexamples to a small number (less than a trillion).  A simple
  computer program that implements Pipletti's classification theorem
  for torsion-free Aramaic groups with simplectic socles can then
  finish the remaining cases.
\par
}

\vskip 12pt plus 4pt minus 4pt

\vskip 0pt plus .3 \vsize
\penalty -250
\vskip 0pt plus -.3 \vsize
\vskip 12pt plus 4pt minus 4pt
\vskip 0pt plus 1pt
\hbox to \hsize {\bold 1. Introduction \hss}
{ \penalty 10000 }
\vskip 3pt plus 1pt minus 1pt
\noindent

This is the start of the introduction.
%{
%\par
%\vfill
%\par
%\penalty -20000
%}
\end
